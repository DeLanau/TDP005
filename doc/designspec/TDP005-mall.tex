\documentclass{TDP005mall}



\newcommand{\version}{Version 1.1}
\author{Elliot Johansson, \url{elljo130@student.liu.se}\\
  Lukas Freyland, \url{lukfr510@student.liu.se}\\
  Nadim Lakrouz, \url{nadla777@student.liu.se}}
\title{Designspecifikation}
\date{2022-11-24}
\rhead{Elliot Johansson\\
Lukas Freyland\\
Nadim Lakrouz}



\begin{document}
\projectpage
\section{Revisionshistorik}
\begin{table}[!h]
\begin{tabularx}{\linewidth}{|l|X|l|}
\hline
Ver. & Revisionsbeskrivning & Datum \\\hline
1.1 & Modifierad för att stödja xelatex och unicode & 150603 \\\hline
1.0 & Skapad för studenter att använda som mall till
kommande dokumentinlämningar & 140908 \\\hline
\end{tabularx}
\end{table}

\section{Player}

Syftet med Playerklassen är att representera den karaktär spelaren styr. 

\subsection{}

\begin{itemize}
    \item int hp - privat variabel som tilldelas ett heltal. 
    \item double mana - privat variabel som tilldelas ett flyttal. 
    \item sf:Clock clock - sfml variabel för att hantera tiden.
    \item int set_hp() - ändrar privata variabeln hp.
    \item int get_hp() - hämtar privata variabeln hp.
    \item int get_mana() - hämtar privata variabeln mana.
    \item void regenerate_mana() - ökar privata variabeln mana över tid.
   
\end{itemize}


\section{Enemies}

\begin{itemize}

    \item double hp - privat variabel som tilldelas ett flyttal. 
    \item double resistance - privat variabel som tilldelas ett flyttal.
    \item void check_collision() - kollar kollision med spelaren och magiska formler. 
    \item int set_hp() - ändrar privata variabeln hp.
    \item int get_hp() - hämtar privata variabeln hp.
    \item int get_resistance() - hämtar privata variabeln resistancce.
    \item void ai_movement() - styr fiende rörelse. 
    
\end{itemize}

\section{Spells_Handler}

\begin{itemize}

    \item pair<string, int> mana_per_spell - lista med hur mycket mana varje magiska formel kostar. 
    \item void handle_input() - hanterar spelarens inmätning. 
    \item void handle_combination() - hanterar kombinationer av magiska     formler. 
    \item void handle_damage() - beräknar och hanterar damage. 
    \item void control_spell() - navigerar spells med hjälp av mus. 
    
\end{itemize}

\end{document}
