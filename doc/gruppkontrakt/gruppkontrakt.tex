\documentclass{mall}

\newcommand{\version}{Version 1.0}
\author{Elliot Johansson, \url{elljo130@student.liu.se}\\
  Nadim Lakrouz, \url{nadla777@student.liu.se}\\
  Lukas Freyland, \url{lukfr510@student.liu.se}}
\title{Gruppkontrakt}
\date{2022-11-10}
\rhead{}


\begin{document}
\projectpage

\section{Förutsättningar}
\label{prereq}



\begin{itemize}
\item \textbf{Följande saker vill jag att min/mina kollegor visar hänsyn och förståelse för}

Vi ska ta hänsyn till allas åsikter samt ha bra kommunikation mellan oss. 

\item \textbf{Hur ska jag bete mig för att stötta min/mina kollegor utifrån sina förutsättningar?}

Se till att alla förstår innehållet. Ta hänsyn till vad alla satsar på betygsmässigt samt att alla i gruppen ska vara trevliga mot varandra.   

\end{itemize}

\section{Hur vi arbetar tillsammans}

\begin{itemize}
\item \textbf{Vilka tider arbetar vi, och vilka tider är vi nåbara utöver detta?}

Vi arbetar mellan tiderna 8-17. På söndagar har vi ett möte där vi går igenom saker vi har gjort den veckan samt de saker som behöver göras till kommande vecka. Om vi känner att det behövs kan vi även ha ett till möte i mitten av veckan för att stämma av arbetet.

\item \textbf{Hur kommunicerar vi med varandra? Vilka verktyg/kanaler använder vi? Hur och när är det okej att vi avbryter varandra?}

Vi använder Discord för att kommunicera med varandra. Mellan tiderna 10:00 och 22:00 kan vi kontakta varandra och förväntas att få svar men om det är sent får vänta på svar tills dagen efter.

\item \textbf{Hur gör vi för att ge varandra möjlighet att framföra åsikter och tankar om uppgifter och idéer till arbetet?}

Vi skapar en säker och öppen arbetsmiljö där alla är bekvämma med att yttra åsikter och tankar.

\item \textbf{Hur ofta tar vi paus? Ska vi hjälpas åt att påminna varandra om att ta paus?}

På labbtider så är det inte så nödvändigt att ta en paus. Men om vi skulle jobba längre sessioner så kan vi försöka ta en paus varje timme för att hålla humöret på topp. Om nån av oss känner att vi behöver en paus så är det bara att säga till.

\item \textbf{Arbetar vi tillsammans med uppgifter, eller var för sig?}

I början arbetar vi tillsammans i större utsträckning för att alla ska få en överblick över projektet. Senare i projektet om vi känner att arbetet tjänar på det, delar vi upp arbetet.

\item \textbf{Hur bestämmer vi vem som gör vad?}

I våra möten går vi igenom arbetet och vem som utför en del av projektet genom att diskutera vem av oss som känner sig mest bekväm med dem.

\item \textbf{Hur specifierar vi vad som ingår i varje uppgift, och när den är klar?}

Genom en to-do lista kan vi dela upp uppgiften och se tydligt vilka moment som ska ingå. När vi är nöjda med resultatet bockar vi av den delen av uppgiften.

\item \textbf{Hur snabbt förväntar vi oss att en uppgift kan vara klar?}

Beroende på storlek av uppgift kan det variera men uppgiften förväntas vara klart ett par dagar innan mjuk deadline vilket är optimalt och ger tillfälle för att komplettera.

\item \textbf{Hur håller vi reda på att uppgifter vi identifierat inte glöms bort?}

Med en to-do list som vi går igenom varje mötestillfälle.

\end{itemize}

\section{Om jag tycker att något inte fungerar}

\begin{itemize}
\item \textbf{Vad gör vi om någon kommer sent?}

Om någon kommer sent måste den personen meddela det. Det finns alltid risk att något kan hända som gör att man inte kan senanmäla sig men om någon kommer sent upprepade gånger får man ta upp det med personen för att diskutera arbetstider och arbetsbelastning för att se om det går att göra kompromisser.

\item \textbf{Vad gör vi om någon inte slutför sina uppgifter?}

Vi går igenom den uppgiften tillsammans och jobbar i grupp för att lösa uppgiften.

\item \textbf{Vad gör vi om arbetsfördelningen blir ojämn?}

Personen som känner att det är ojämnt tar upp det på ett möte där vi tillsammans får gå igenom uppdelningen av uppgifterna igen.

\item \textbf{Hur tar vi upp ett problem med berörda personer?}

Vi pratar om det och om någon inte är bekväm att tala med den andra personen får man plocka in en tredje part som medlar mellan personerna.

\item \textbf{Hur ger jag kritik och beröm till andra personer i gruppen?}

Se till att ge konstruktiv kritik på ett vänligt sätt. Se till att ge beröm när någon gör något bra.

\end{itemize}

\section{Utvärdering}


\begin{itemize}
\item \textbf{När ska vi påminna oss om gruppkontraktet och utvärdera hur det fungerat?}

Om det märks att arbetet inte fungerar kan vi gå tillbaka till kontraktet och gå igenom det.

\end{itemize}

\end{document}
